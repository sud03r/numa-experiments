
Most modern multicore systems these days are NUMA, which means they have multiple memory comtrollers
with non-uniform access latencies across them. There has been significant amount of work done in exploring and 
mitigating performance penalty due to NUMA overhead. In previous works, NUMA-aware schedulers were proposed,
sometimes with an objective of keeping data close to the processors working on them and sometimes to attain a 
tradeoff between data locality and cache contention. \\
However, the effect of instructions on process' performance in context of NUMA has not been explored yet.
In this work, we present an analysis about the performance hit on processes due to instructions on NUMA machines.
We limit our focus to the impact of shared code, which by definition has a single copy and resides on one of the NUMA nodes.
We also discuss various scenarios in which a single copy of these shared objects can harm the performance due to remote
fetches and present some ideas to alleviate the problem.

