The work done by Brecht \cite{Brecht:1993:IPA:1295480.1295481} explains that in the case of 
NUMA scheduler the question of which processor becomes more important than how many processors to be assigned for a task.
His results have shown thread placement decision is critical in large scale multiprocessors.
Work done by Broquedis et al \cite{numaScheduling} is also on the similar lines and talks about
dynamic task and data placement over NUMA architecture.
In another recent work \cite{Dashti:2013:TMH:2490301.2451157}, the authors claim that contrary to 
older systems, modern NUMA hardware has much smaller remote wire delays, and so remote access
costs should not be the main concern for performance, instead, 
congestion on memory controllers and interconnects, caused by memory traffic from 
data-intensive applications, hurts performance a lot more.
Majo et. al. \cite{Majo:2011:MMN:1993478.1993481} have exploited the trade-off 
between cache contention and interconnect overhead for their NUMA aware scheduler.

Apart from these, there also have been some work concerning user-text replication by the
Linux community. \cite{textReplication} have performed a similar analysis but in context of
system calls, claiming a benefit of about 41\% with cold cache.

